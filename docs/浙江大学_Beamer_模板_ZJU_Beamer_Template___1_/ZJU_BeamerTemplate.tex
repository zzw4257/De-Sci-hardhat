%# -*- coding:utf-8 -*-
\documentclass[10pt,aspectratio=169,mathserif]{beamer}		
%设置为 Beamer 文档类型,设置字体为 10pt,长宽比为16:9,数学字体为 serif 风格

%%%%-----导入宏包-----%%%%
\usepackage{zju}			%导入 zju 模板宏包
\usepackage{ctex}			%导入 ctex 宏包,添加中文支持
\usepackage{amsmath,amsfonts,amssymb,bm}   %导入数学公式所需宏包
\usepackage{color}			 %字体颜色支持
\usepackage{graphicx,hyperref,url}
\usepackage{metalogo}	% 非必须
%% 上文引用的包可按实际情况自行增删
%%%%%%%%%%%%%%%%%%
\usepackage{fontspec}
\usepackage{xeCJK}
% \setCJKmainfont{Source Han Sans SC}



\beamertemplateballitem		%设置 Beamer 主题

%%%%------------------------%%%%%
\catcode`\。=\active         %或者=13
\newcommand{。}{.}				
%将正文中的“。”号转换为“.”。中文标点国家规范建议科技文献中的句号用圆点替代
%%%%%%%%%%%%%%%%%%%%%

%%%%----首页信息设置----%%%%
\title[浙江大学 Beamer 模板]{浙江大学 Beamer 模板}
\subtitle{——这里是副标题}			
%%%%----标题设置


\author[Guang Touge]{
  光头哥 \\\medskip
  {\small \url{chenqiyuan1012@foxmail.com}} \\
  {\small \url{https://www.zju.edu.cn/}}}
%%%%----个人信息设置
  
\institute[IOPP]{
  计算机学院 \\ 
  浙江大学}
%%%%----机构信息

\date[Aug. 31 2023]{
  2023年8月31日}
%%%%----日期信息
  
\begin{document}

\begin{frame}
	\titlepage
\end{frame}				%生成标题页

\section{提纲}
\begin{frame}
	\frametitle{提纲}
	\tableofcontents
\end{frame}				%生成提纲页

\section{介绍}
\begin{frame}
	\frametitle{介绍}

	\begin{itemize}
		\item {编译方式}
		      \begin{itemize}
			      \item  推荐安装完整版的 TeXLive
			      \item 使用 \XeLaTeX 编译
		      \end{itemize}
		\item 请参考 \LaTeX 和 Beamer 用户文档

		\item 行内数学公式示例 $\sin^2 \theta + \cos^2 \theta = 1$
		\item {行间数学公式示例 \begin{equation}
			      y_{1}=\int \sin x\, {\rm d}x
		      \end{equation}	 }
		\item 基于“浙大蓝”颜色 \url{https://www.zju.edu.cn/}
	\end{itemize}
\end{frame}

\section{内置环境}
\begin{frame}
	\frametitle{内置环境}
	\begin{block}{Slides with \LaTeX}
		Beamer offers a lot of functions to create nice slides using \LaTeX.
	\end{block}

	\begin{block}{The basis}
		内部使用以下主题
		\begin{itemize}
			\item split
			\item whale
			\item rounded
			\item orchid
		\end{itemize}
	\end{block}
\end{frame}

\begin{frame}
	\frametitle{带数字列表}
	\begin{enumerate}
		\item This just shows the effect of the style
		\item It is not a Beamer tutorial
		\item Read the Beamer manual for more help
		\item Contact me only concerning the style file
	\end{enumerate}
\end{frame}

\section{DeSci平台端到端演示}
\begin{frame}
	\frametitle{完整演示流程 - 链上写链下读+数据库集成}
	
	\begin{columns}[c]
		\column{0.5\textwidth}
		\begin{block}{演示步骤}
			\begin{enumerate}
				\item \small \textbf{启动环境} - 部署10个合约到Hardhat
				\item \small \textbf{后端服务} - Go服务连接SQLite+事件监听
				\item \small \textbf{链上交互} - 用户注册、发表研究成果NFT
				\item \small \textbf{链下同步} - 捕获事件并入库SQLite
				\item \small \textbf{数据库验证} - 查询验证数据完整性
				\item \small \textbf{API验证} - RESTful接口获取数据
			\end{enumerate}
		\end{block}
		
		\column{0.5\textwidth}
		\begin{center}
			% 端到端架构图留白
			\fbox{\parbox{0.9\textwidth}{\centering
				\vspace{3cm}
				{\Large \textcolor{blue}{[端到端架构图]}}
				\vspace{0.5cm}
				
				\small 智能合约 → 事件监听 → SQLite → API
				\vspace{0.5cm}
			}}
		\end{center}
	\end{columns}
	
	\vspace{0.3cm}
	\begin{alertblock}{演示重点}
		完整展示链上写、链下读与数据库真实集成的DeSci平台
	\end{alertblock}
\end{frame}

\begin{frame}
	\frametitle{步骤1: 环境启动与合约部署 🚀}
	
	\begin{columns}[c]
		\column{0.5\textwidth}
		\begin{block}{启动命令}
			\texttt{\$ npm run start-hardhat}\\
			\texttt{\$ npm run deploy-contracts}
		\end{block}
		
		\begin{block}{验证要点}
			\begin{itemize}
				\item Hardhat网络 (Chain ID: 31337) 启动成功
				\item 10个智能合约全部部署完成
				\item 合约地址输出并记录到配置文件
			\end{itemize}
		\end{block}
		
		\column{0.5\textwidth}
		% 环境启动截图留白
		\begin{center}
			\fbox{\parbox{0.9\textwidth}{\centering
				\vspace{2.5cm}
				{\Large \textcolor{green}{[Hardhat部署日志]}}
				\vspace{0.5cm}
				
				\small
				\texttt{✅ DeSciRegistry: 0x5FbDB...\\
				✅ DatasetManager: 0xe7f17...\\
				✅ ResearchNFT: 0x9fE46...\\
				...10个合约部署完成}
				\vspace{0.5cm}
			}}
		\end{center}
	\end{columns}
\end{frame}

\begin{frame}
	\frametitle{步骤2: 后端服务启动与数据库连接 💾}
	
	\begin{block}{启动命令}
		\texttt{\$ cd backend \&\& go run cmd/server/main.go}
	\end{block}
	
	\begin{columns}[c]
		\column{0.5\textwidth}
		\begin{block}{验证要点}
			\begin{itemize}
				\item SQLite数据库连接成功
				\item 区块链事件监听器启动
				\item HTTP API服务器运行 (端口8080)
				\item 合约地址配置加载完成
			\end{itemize}
		\end{block}
		
		\column{0.5\textwidth}
		% 后端启动日志截图留白
		\begin{center}
			\fbox{\parbox{0.9\textwidth}{\centering
				\vspace{2.5cm}
				{\Large \textcolor{blue}{[Go后端启动日志]}}
				\vspace{0.5cm}
				
				\small 
				\texttt{✅ SQLite connected\\
				🔗 Event listeners started\\
				🚀 Server running on :8080}
				\vspace{1cm}
			}}
		\end{center}
	\end{columns}
\end{frame}

\begin{frame}
	\frametitle{步骤3: 链上交互与事件触发 ⛓️}
	
	\begin{columns}[c]
		\column{0.5\textwidth}
		\begin{block}{演示脚本}
			\texttt{\$ npx hardhat run scripts/\\
			fullTestScenario.js --network localhost}
		\end{block}
		
		\begin{block}{模拟操作}
			\begin{itemize}
				\item 用户注册 → UserRegistered事件
				\item 上传数据集 → DatasetUploaded事件
				\item 发表研究成果 → ResearchMinted事件
			\end{itemize}
		\end{block}
		
		\column{0.5\textwidth}
		% 链上交互截图留白
		\begin{center}
			\fbox{\parbox{0.9\textwidth}{\centering
				\vspace{2.5cm}
				{\Large \textcolor{orange}{[Hardhat交易日志]}}
				\vspace{0.5cm}
				
				\small 
				\texttt{Transaction: 0xabc123...\\
				ResearchMinted(tokenId: 1)\\
				Gas Used: 234,567}
				\vspace{1cm}
			}}
		\end{center}
	\end{columns}
\end{frame}

\begin{frame}
	\frametitle{步骤4: 链下事件同步与数据库写入 🔄}
	
	\begin{block}{事件监听验证}
		Go后端实时捕获链上事件并入库SQLite数据库
	\end{block}
	
	\begin{columns}[c]
		\column{0.5\textwidth}
		\begin{block}{同步过程}
			\begin{itemize}
				\item 监听到ResearchMinted事件
				\item 提取事件参数:tokenId, author, contentHash
				\item 插入到research\_data表
				\item 记录区块号和交易哈希
			\end{itemize}
		\end{block}
		
		\column{0.5\textwidth}
		% 事件同步日志截图留白
		\begin{center}
			\fbox{\parbox{0.9\textwidth}{\centering
				\vspace{2.5cm}
				{\Large \textcolor{purple}{[事件同步日志]}}
				\vspace{0.5cm}
				
				\small 
				\texttt{📡 Event Captured: ResearchMinted\\
				💾 Inserting to SQLite...\\
				✅ Database sync completed}
				\vspace{1cm}
			}}
		\end{center}
	\end{columns}
\end{frame}

\begin{frame}
	\frametitle{步骤5: SQLite数据库验证 🗃️}
	
	\begin{columns}[c]
		\column{0.5\textwidth}
		\begin{block}{数据库查询}
			直接查询SQLite验证数据完整性
		\end{block}
		
		\begin{alertblock}{SQL查询命令}
			\texttt{SELECT * FROM research\_data\\
			WHERE token\_id = '1';}
		\end{alertblock}
		
		\column{0.5\textwidth}
		% 数据库查询截图留白
		\begin{center}
			\fbox{\parbox{0.9\textwidth}{\centering
				\vspace{2.5cm}
				{\Large \textcolor{green}{[SQLite查询结果]}}
				\vspace{0.5cm}
				
				\small
				\texttt{token\_id: 1\\
				title: 区块链科研应用\\
				authors: ["0x123..."]\\
				block\_number: 12345\\
				tx\_hash: 0xabc123...}
				\vspace{0.5cm}
			}}
		\end{center}
	\end{columns}
	
	\vspace{0.3cm}
	\begin{block}{验证要点}
		✅ 链上数据与数据库数据完全一致 \quad ✅ 可追溯性:包含区块号和交易哈希
	\end{block}
\end{frame}

\begin{frame}
	\frametitle{步骤6: RESTful API验证与数据一致性 📡}
	
	\begin{columns}[c]
		\column{0.5\textwidth}
		\begin{block}{API测试}
			通过REST API获取刚才上链的数据
		\end{block}
		
		\begin{alertblock}{测试命令}
			\texttt{curl -X GET \\
			"http://localhost:8080/api/v1/\\
			research/1"}
		\end{alertblock}
		
		\column{0.5\textwidth}
		% API验证截图留白
		\begin{center}
			\fbox{\parbox{0.9\textwidth}{\centering
				\vspace{2.5cm}
				{\Large \textcolor{cyan}{[API响应结果]}}
				\vspace{0.5cm}
				
				\small
				\texttt{\{\\
				"tokenId": "1",\\
				"title": "区块链科研应用",\\
				"authors": ["0x123..."],\\
				"contentHash": "0xabc...",\\
				"blockNumber": 12345\\
				\}}
				\vspace{0.5cm}
			}}
		\end{center}
	\end{columns}
	
	\vspace{0.3cm}
	\begin{block}{一致性验证}
		✅ API数据 = SQLite数据 = 链上数据 \quad ✅ 端到端数据完整性验证成功
	\end{block}
\end{frame}

\begin{frame}
	\frametitle{演示总结 - 数据库真实集成验证 🎯}
	
	\begin{columns}[c]
		\column{0.6\textwidth}
		\begin{block}{演示完成验证清单}
			\begin{itemize}
				\item ✅ \textbf{环境部署} - 10个智能合约成功部署
				\item ✅ \textbf{数据库连接} - SQLite真实连接并建表
				\item ✅ \textbf{链上交互} - 用户注册、研究成果发表
				\item ✅ \textbf{事件同步} - 实时捕获并入库SQLite
				\item ✅ \textbf{数据一致性} - 链上链下数据完全一致
				\item ✅ \textbf{API验证} - RESTful接口获取数据库数据
			\end{itemize}
		\end{block}
		
		\begin{alertblock}{核心价值体现}
			完整展示了\textbf{链上写、链下读与数据库真实集成}的DeSci平台架构!
		\end{alertblock}
		
		\column{0.4\textwidth}
		% 数据库集成架构图留白
		\begin{center}
			\fbox{\parbox{0.9\textwidth}{\centering
				\vspace{3cm}
				{\Large \textcolor{red}{[数据库集成架构]}}
				\vspace{0.5cm}
				
				\small 智能合约 → 事件监听 → SQLite → API
				\vspace{1cm}
			}}
		\end{center}
	\end{columns}
\end{frame}

\section{结论}
\begin{frame}
	\frametitle{结论}

	\begin{itemize}
		\item Easy to use
		\item Good results
	\end{itemize}
\end{frame}

\section{参考文献}
\begin{frame}{参考文献}
	\begin{thebibliography}{99}
		\bibitem{zhao1} Yi~Zhao, {\sl An introduction to X}, Sep.~15, 2015
		\bibitem{qian2} Er~Qian, San~Sun,
		Phys.\ Lett.\ A {\bf xx}, 2xx (20xx)
		\bibitem{li4} Si~Li, Phys.\ Rev.\ C {\bf xx}, 5xx (20xx)

	\end{thebibliography}
\end{frame}

\end{document}