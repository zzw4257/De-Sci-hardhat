% !TEX program = xelatex
\documentclass[10pt]{beamer}
\usetheme{Madrid}
\usecolortheme{seahorse}
\usepackage[UTF8]{ctex}
\usepackage{graphicx}
\usepackage{listings}
\usepackage{xcolor}
\usepackage{amsmath}
\title{DeSci 链上写 / 链下读 MVP 架构展示}
\subtitle{高信息密度 · 技术亮点 · 8-10 分钟}
\author{周子为 \and 张家畅}
\date{\\(内部讨论稿)}

\begin{document}

% 标题页
\begin{frame}
  \titlepage
\end{frame}

% 目录
\begin{frame}{目录}
  \tableofcontents
\end{frame}

\section{问题与动机}
\begin{frame}{痛点速览}
  % 列表:使用五个关键词占位
  % 【高Gas】【查询慢】【链上存储冗余】【不可扩展】【数据无法验证】
\end{frame}
\begin{frame}{Why 链上写 / 链下读}
  % 对比图占位:左侧“全部链上” 右侧“MVP架构”
  % 图片占位描述:图1:两列结构对比(标注Gas/延迟)
\end{frame}

\section{总体架构}
\begin{frame}{设计原则}
  % 三列占位:【最小可用】【可验证完整性】【可向复杂扩展】
  【最小可用】——只含事件监听/入库/查询/验证\\
  【可验证完整性】——链上哈希对照链下内容\\
  【可向复杂扩展】——预留 Hook(缓存/指标/批量校验)
\end{frame}
\begin{frame}{架构总览}
  % 图片:事件驱动数据流
  % 图片占位描述:用户 -> 合约(事件) -> Listener -> DB -> API -> 校验接口
\end{frame}

\section{链上设计}
\begin{frame}{合约最小交互面}
  【ResearchNFT.Created】核心新增\\
  【ResearchNFT.Updated】元数据演化\\
  【DatasetManager.Created】引用/聚合入口\\
  % 说明:仅取高价值事件,减少监听复杂度
\end{frame}
\begin{frame}{事件模型}
  % 表格占位:事件名 | 主键字段 | 关键参数
  【事件名列】:Created / Updated / DatasetCreated\\
  【主键字段】:tokenId / datasetId\\
  【关键参数】:author / dataHash / blockTimestamp\\
  【去重键】:txHash + logIndex\\
  % 图片占位:事件结构框图(框内字段分组:身份/内容哈希/链位置)
\end{frame}
\begin{frame}{哈希策略}
  % 步骤占位:获取明文 -> keccak256 -> 比对 -> 返回match
  【Step1】接收明文(raw)\\
  【Step2】UTF8标准化\\
  【Step3】keccak256(raw) -> localHash\\
  【Step4】对比链上/DB dataHash\\
  【Result】match / mismatch\\
  % 图片占位:Hash 验证流程箭头图(左 raw,中 hash,右 判定)
\end{frame}

\section{链下设计}
\begin{frame}{事件到入库链路}
  % 流程节点占位:Subscribe -> Decode -> Map -> Dedup -> Insert
  % 图片占位:时序图
\end{frame}
\begin{frame}{最小数据模型}
  % 表格占位:research_data / event_log 字段列表
  % 图片占位:ER简图
\end{frame}
\begin{frame}{暂不做的(控制范围)}
  % 列表:【缓存】【指标】【批量校验】【统计】 -> 标注“Hook 预留”
\end{frame}

\section{关键流程}
\begin{frame}{写入流程(事件驱动)}
  % 步骤占位:1 Mint -> 2 Emit -> 3 Listen -> 4 Persist -> 5 可查询
  【1 用户调用 Mint】\\
  【2 合约 Emit Created】\\
  【3 Listener 订阅解码】\\
  【4 去重+映射后入库】\\
  【5 API 可查询】\\
  % 图片:流程箭头(上方主链路,下方注释“幂等点”)
\end{frame}
\begin{frame}{查询与验证}
  % 两列:左“列表&作者过滤” 右“单条哈希校验”
  % 图片:API调用示意
\end{frame}
\begin{frame}{失败与降级}
  % 列表:解析失败->日志; 写入冲突->忽略; 校验失败->返回 mismatch
  【解析失败】-> 记录 error + 跳过\\
  【JSON 映射异常】-> 标注 payloadRaw\\
  【写入冲突】(已存在) -> 忽略(幂等)\\
  【校验失败】-> 返回 mismatch(不阻断查询)\\
  【缺失字段】-> 占位值 ("Untitled")\\
  【未知事件】-> 忽略 + debug log
\end{frame}

\section{扩展预留}
\begin{frame}{幂等与去重}
  % 核心:txHash + logIndex
  % 说明:冲突=已处理
  【去重键】txHash+logIndex 复合唯一约束\\
  【重复事件】直接跳过不回滚\\
  【乱序到达】依赖区块号排序视图\\
  【起始区块】START\_BLOCK 手动配置\\
  【重放场景】可停机修改 START\_BLOCK\\
  【失败重试】当前批次跳过,不阻塞主循环
\end{frame}
\begin{frame}{扩展挂钩位}
  % 表格:缓存 | 指标 | 重放 | 批量校验 | 权限
  % 每项写“Hook”占位
  【缓存 Hook】service 查询返回处\\
  【指标 Hook】listener 事件成功/失败计数\\
  【重放 Hook】启动阶段 scan(range)\\
  【批量校验 Hook】verify 包预留 batch 接口\\
  【权限 Hook】API 入口包装层\\
  【审计 Hook】写入成功后追加日志队列
\end{frame}

\section{演示规划}
\begin{frame}{Demo 剧本}
  % 步骤占位:1 health -> 2 mint -> 3 latest -> 4 verify match -> 5 verify mismatch
  % 图片:终端+API对比示意
\end{frame}
\begin{frame}{角色分工}
  % 左:周子为(事件 & 哈希) 右:张家畅(API & 数据)
  % 图片:责任矩阵占位
\end{frame}

\section{总结}
\begin{frame}{价值 \& 下一步}
  % 三收益占位:【Gas 降】【查询快】【可验证】
  % 两下一步占位:【缓存层引入】【批量校验】
\end{frame}

\end{document}
